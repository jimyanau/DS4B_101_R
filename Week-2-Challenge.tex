\documentclass[]{article}
\usepackage{lmodern}
\usepackage{amssymb,amsmath}
\usepackage{ifxetex,ifluatex}
\usepackage{fixltx2e} % provides \textsubscript
\ifnum 0\ifxetex 1\fi\ifluatex 1\fi=0 % if pdftex
  \usepackage[T1]{fontenc}
  \usepackage[utf8]{inputenc}
\else % if luatex or xelatex
  \ifxetex
    \usepackage{mathspec}
  \else
    \usepackage{fontspec}
  \fi
  \defaultfontfeatures{Ligatures=TeX,Scale=MatchLowercase}
\fi
% use upquote if available, for straight quotes in verbatim environments
\IfFileExists{upquote.sty}{\usepackage{upquote}}{}
% use microtype if available
\IfFileExists{microtype.sty}{%
\usepackage{microtype}
\UseMicrotypeSet[protrusion]{basicmath} % disable protrusion for tt fonts
}{}
\usepackage[margin=1in]{geometry}
\usepackage{hyperref}
\hypersetup{unicode=true,
            pdftitle={Week 2 Challenge},
            pdfauthor={Business Science},
            pdfborder={0 0 0},
            breaklinks=true}
\urlstyle{same}  % don't use monospace font for urls
\usepackage{color}
\usepackage{fancyvrb}
\newcommand{\VerbBar}{|}
\newcommand{\VERB}{\Verb[commandchars=\\\{\}]}
\DefineVerbatimEnvironment{Highlighting}{Verbatim}{commandchars=\\\{\}}
% Add ',fontsize=\small' for more characters per line
\usepackage{framed}
\definecolor{shadecolor}{RGB}{248,248,248}
\newenvironment{Shaded}{\begin{snugshade}}{\end{snugshade}}
\newcommand{\AlertTok}[1]{\textcolor[rgb]{0.94,0.16,0.16}{#1}}
\newcommand{\AnnotationTok}[1]{\textcolor[rgb]{0.56,0.35,0.01}{\textbf{\textit{#1}}}}
\newcommand{\AttributeTok}[1]{\textcolor[rgb]{0.77,0.63,0.00}{#1}}
\newcommand{\BaseNTok}[1]{\textcolor[rgb]{0.00,0.00,0.81}{#1}}
\newcommand{\BuiltInTok}[1]{#1}
\newcommand{\CharTok}[1]{\textcolor[rgb]{0.31,0.60,0.02}{#1}}
\newcommand{\CommentTok}[1]{\textcolor[rgb]{0.56,0.35,0.01}{\textit{#1}}}
\newcommand{\CommentVarTok}[1]{\textcolor[rgb]{0.56,0.35,0.01}{\textbf{\textit{#1}}}}
\newcommand{\ConstantTok}[1]{\textcolor[rgb]{0.00,0.00,0.00}{#1}}
\newcommand{\ControlFlowTok}[1]{\textcolor[rgb]{0.13,0.29,0.53}{\textbf{#1}}}
\newcommand{\DataTypeTok}[1]{\textcolor[rgb]{0.13,0.29,0.53}{#1}}
\newcommand{\DecValTok}[1]{\textcolor[rgb]{0.00,0.00,0.81}{#1}}
\newcommand{\DocumentationTok}[1]{\textcolor[rgb]{0.56,0.35,0.01}{\textbf{\textit{#1}}}}
\newcommand{\ErrorTok}[1]{\textcolor[rgb]{0.64,0.00,0.00}{\textbf{#1}}}
\newcommand{\ExtensionTok}[1]{#1}
\newcommand{\FloatTok}[1]{\textcolor[rgb]{0.00,0.00,0.81}{#1}}
\newcommand{\FunctionTok}[1]{\textcolor[rgb]{0.00,0.00,0.00}{#1}}
\newcommand{\ImportTok}[1]{#1}
\newcommand{\InformationTok}[1]{\textcolor[rgb]{0.56,0.35,0.01}{\textbf{\textit{#1}}}}
\newcommand{\KeywordTok}[1]{\textcolor[rgb]{0.13,0.29,0.53}{\textbf{#1}}}
\newcommand{\NormalTok}[1]{#1}
\newcommand{\OperatorTok}[1]{\textcolor[rgb]{0.81,0.36,0.00}{\textbf{#1}}}
\newcommand{\OtherTok}[1]{\textcolor[rgb]{0.56,0.35,0.01}{#1}}
\newcommand{\PreprocessorTok}[1]{\textcolor[rgb]{0.56,0.35,0.01}{\textit{#1}}}
\newcommand{\RegionMarkerTok}[1]{#1}
\newcommand{\SpecialCharTok}[1]{\textcolor[rgb]{0.00,0.00,0.00}{#1}}
\newcommand{\SpecialStringTok}[1]{\textcolor[rgb]{0.31,0.60,0.02}{#1}}
\newcommand{\StringTok}[1]{\textcolor[rgb]{0.31,0.60,0.02}{#1}}
\newcommand{\VariableTok}[1]{\textcolor[rgb]{0.00,0.00,0.00}{#1}}
\newcommand{\VerbatimStringTok}[1]{\textcolor[rgb]{0.31,0.60,0.02}{#1}}
\newcommand{\WarningTok}[1]{\textcolor[rgb]{0.56,0.35,0.01}{\textbf{\textit{#1}}}}
\usepackage{graphicx,grffile}
\makeatletter
\def\maxwidth{\ifdim\Gin@nat@width>\linewidth\linewidth\else\Gin@nat@width\fi}
\def\maxheight{\ifdim\Gin@nat@height>\textheight\textheight\else\Gin@nat@height\fi}
\makeatother
% Scale images if necessary, so that they will not overflow the page
% margins by default, and it is still possible to overwrite the defaults
% using explicit options in \includegraphics[width, height, ...]{}
\setkeys{Gin}{width=\maxwidth,height=\maxheight,keepaspectratio}
\IfFileExists{parskip.sty}{%
\usepackage{parskip}
}{% else
\setlength{\parindent}{0pt}
\setlength{\parskip}{6pt plus 2pt minus 1pt}
}
\setlength{\emergencystretch}{3em}  % prevent overfull lines
\providecommand{\tightlist}{%
  \setlength{\itemsep}{0pt}\setlength{\parskip}{0pt}}
\setcounter{secnumdepth}{0}
% Redefines (sub)paragraphs to behave more like sections
\ifx\paragraph\undefined\else
\let\oldparagraph\paragraph
\renewcommand{\paragraph}[1]{\oldparagraph{#1}\mbox{}}
\fi
\ifx\subparagraph\undefined\else
\let\oldsubparagraph\subparagraph
\renewcommand{\subparagraph}[1]{\oldsubparagraph{#1}\mbox{}}
\fi

%%% Use protect on footnotes to avoid problems with footnotes in titles
\let\rmarkdownfootnote\footnote%
\def\footnote{\protect\rmarkdownfootnote}

%%% Change title format to be more compact
\usepackage{titling}

% Create subtitle command for use in maketitle
\providecommand{\subtitle}[1]{
  \posttitle{
    \begin{center}\large#1\end{center}
    }
}

\setlength{\droptitle}{-2em}

  \title{Week 2 Challenge}
    \pretitle{\vspace{\droptitle}\centering\huge}
  \posttitle{\par}
    \author{Business Science}
    \preauthor{\centering\large\emph}
  \postauthor{\par}
      \predate{\centering\large\emph}
  \postdate{\par}
    \date{12/30/2018}


\begin{document}
\maketitle

{
\setcounter{tocdepth}{2}
\tableofcontents
}
\hypertarget{challenge-summary}{%
\section{Challenge Summary}\label{challenge-summary}}

This is a short challenge to begin applying what you are learning to the
problem at hand. You will go through a series of questions related to
the course project goals:

\begin{enumerate}
\def\labelenumi{\arabic{enumi}.}
\item
  Coming up with a new product idea, and
\item
  Segmenting the customer-base
\end{enumerate}

\hypertarget{objectives}{%
\section{Objectives}\label{objectives}}

\begin{enumerate}
\def\labelenumi{\arabic{enumi}.}
\item
  Apply \texttt{dplyr} and \texttt{tidyr} functions to answer questions
  related to the course projects.
\item
  Gain exposure to \texttt{rmarkdown}
\end{enumerate}

\hypertarget{data}{%
\section{Data}\label{data}}

To read the data, make sure that the paths point to the appropriate data
sets. Saving the file in the main directory should enable the paths to
be detected correctly.

\begin{Shaded}
\begin{Highlighting}[]
\CommentTok{# Load libraries}
\KeywordTok{library}\NormalTok{(tidyverse)}
\end{Highlighting}
\end{Shaded}

\begin{Shaded}
\begin{Highlighting}[]
\CommentTok{# Read bike orderlines data}
\NormalTok{path_bike_orderlines <-}\StringTok{ "00_data/bike_sales/data_wrangled/bike_orderlines.rds"}
\NormalTok{bike_orderlines_tbl <-}\StringTok{ }\KeywordTok{read_rds}\NormalTok{(path_bike_orderlines)}

\KeywordTok{glimpse}\NormalTok{(bike_orderlines_tbl)}
\end{Highlighting}
\end{Shaded}

\begin{verbatim}
## Observations: 15,644
## Variables: 13
## $ order_date     <dttm> 2011-01-07, 2011-01-07, 2011-01-10, 2011-01-10...
## $ order_id       <dbl> 1, 1, 2, 2, 3, 3, 3, 3, 3, 4, 5, 5, 5, 5, 6, 6,...
## $ order_line     <dbl> 1, 2, 1, 2, 1, 2, 3, 4, 5, 1, 1, 2, 3, 4, 1, 2,...
## $ quantity       <dbl> 1, 1, 1, 1, 1, 1, 1, 1, 1, 1, 1, 2, 1, 1, 1, 1,...
## $ price          <dbl> 6070, 5970, 2770, 5970, 10660, 3200, 12790, 533...
## $ total_price    <dbl> 6070, 5970, 2770, 5970, 10660, 3200, 12790, 533...
## $ model          <chr> "Jekyll Carbon 2", "Trigger Carbon 2", "Beast o...
## $ category_1     <chr> "Mountain", "Mountain", "Mountain", "Mountain",...
## $ category_2     <chr> "Over Mountain", "Over Mountain", "Trail", "Ove...
## $ frame_material <chr> "Carbon", "Carbon", "Aluminum", "Carbon", "Carb...
## $ bikeshop_name  <chr> "Ithaca Mountain Climbers", "Ithaca Mountain Cl...
## $ city           <chr> "Ithaca", "Ithaca", "Kansas City", "Kansas City...
## $ state          <chr> "NY", "NY", "KS", "KS", "KY", "KY", "KY", "KY",...
\end{verbatim}

\begin{Shaded}
\begin{Highlighting}[]
\CommentTok{# Read bikes data}
\NormalTok{path_bikes <-}\StringTok{ "00_data/bike_sales//data_raw/bikes.xlsx"}
\NormalTok{bikes_tbl <-}\StringTok{ }\NormalTok{readxl}\OperatorTok{::}\KeywordTok{read_excel}\NormalTok{(path_bikes)}

\KeywordTok{glimpse}\NormalTok{(bikes_tbl)}
\end{Highlighting}
\end{Shaded}

\begin{verbatim}
## Observations: 97
## Variables: 4
## $ bike.id     <dbl> 1, 2, 3, 4, 5, 6, 7, 8, 9, 10, 11, 12, 13, 14, 15,...
## $ model       <chr> "Supersix Evo Black Inc.", "Supersix Evo Hi-Mod Te...
## $ description <chr> "Road - Elite Road - Carbon", "Road - Elite Road -...
## $ price       <dbl> 12790, 10660, 7990, 5330, 4260, 3940, 3200, 2660, ...
\end{verbatim}

\# Questions

\hypertarget{what-are-the-unique-categories-of-products-difficulty-low}{%
\subsection{1. What are the unique categories of products? (Difficulty =
Low)}\label{what-are-the-unique-categories-of-products-difficulty-low}}

\begin{itemize}
\tightlist
\item
  Begin with \texttt{bike\_orderlines\_tbl}
\item
  Use \texttt{distinct()} to evaluate
\end{itemize}

Review Primary Product Category (\texttt{category\_1}).

\begin{Shaded}
\begin{Highlighting}[]
\NormalTok{bike_orderlines_tbl }\OperatorTok
\StringTok{    }\KeywordTok{select}\NormalTok{(category_}\DecValTok{1}\NormalTok{) }\OperatorTok
\StringTok{    }\KeywordTok{set_names}\NormalTok{(}\StringTok{"Primary Category"}\NormalTok{) }\OperatorTok
\StringTok{    }\KeywordTok{unique}\NormalTok{() }\OperatorTok
\StringTok{    }\KeywordTok{arrange}\NormalTok{(}\StringTok{`}\DataTypeTok{Primary Category}\StringTok{`}\NormalTok{)}
\end{Highlighting}
\end{Shaded}

\begin{verbatim}
## # A tibble: 2 x 1
##   `Primary Category`
##   <chr>             
## 1 Mountain          
## 2 Road
\end{verbatim}

Review Secondary Product Category (\texttt{category\_2}).

\begin{Shaded}
\begin{Highlighting}[]
\NormalTok{bike_orderlines_tbl }\OperatorTok
\StringTok{    }\KeywordTok{select}\NormalTok{(category_}\DecValTok{2}\NormalTok{) }\OperatorTok
\StringTok{    }\KeywordTok{set_names}\NormalTok{(}\StringTok{"Secondary Category"}\NormalTok{) }\OperatorTok
\StringTok{    }\KeywordTok{unique}\NormalTok{() }\OperatorTok
\StringTok{    }\KeywordTok{arrange}\NormalTok{(}\StringTok{`}\DataTypeTok{Secondary Category}\StringTok{`}\NormalTok{)}
\end{Highlighting}
\end{Shaded}

\begin{verbatim}
## # A tibble: 9 x 1
##   `Secondary Category`
##   <chr>               
## 1 Cross Country Race  
## 2 Cyclocross          
## 3 Elite Road          
## 4 Endurance Road      
## 5 Fat Bike            
## 6 Over Mountain       
## 7 Sport               
## 8 Trail               
## 9 Triathalon
\end{verbatim}

Review Frame Material (\texttt{frame\_material}).

\begin{Shaded}
\begin{Highlighting}[]
\NormalTok{bike_orderlines_tbl }\OperatorTok
\StringTok{    }\KeywordTok{select}\NormalTok{(frame_material) }\OperatorTok
\StringTok{    }\KeywordTok{unique}\NormalTok{() }\OperatorTok
\StringTok{    }\KeywordTok{arrange}\NormalTok{(frame_material)}
\end{Highlighting}
\end{Shaded}

\begin{verbatim}
## # A tibble: 2 x 1
##   frame_material
##   <chr>         
## 1 Aluminum      
## 2 Carbon
\end{verbatim}

\hypertarget{which-product-categories-have-the-most-sales-difficulty-medium}{%
\subsection{2. Which product categories have the most sales? (Difficulty
=
Medium)}\label{which-product-categories-have-the-most-sales-difficulty-medium}}

\begin{itemize}
\tightlist
\item
  Select appropriate columns from \texttt{bike\_orderlines\_tbl}
\item
  Group and summarize the data calling the new column \texttt{Sales}.
  Make sure to ungroup.
\item
  Arrange descending by \texttt{Sales}
\item
  Rename column names to \texttt{Primary\ Category},
  \texttt{Secondary\ Category}, or \texttt{Frame\ Material} (as
  appropriate).
\item
  Format the Sales as \texttt{dollar()}
\end{itemize}

Review Primary Product Category (\texttt{category\_1}).

\begin{Shaded}
\begin{Highlighting}[]
\NormalTok{bike_orderlines_tbl }\OperatorTok
\StringTok{    }\KeywordTok{select}\NormalTok{(category_}\DecValTok{1}\NormalTok{, total_price) }\OperatorTok
\StringTok{    }\KeywordTok{group_by}\NormalTok{(category_}\DecValTok{1}\NormalTok{) }\OperatorTok
\StringTok{    }\KeywordTok{summarise}\NormalTok{(}
        \DataTypeTok{sales =} \KeywordTok{sum}\NormalTok{(total_price)}
\NormalTok{    ) }\OperatorTok
\StringTok{    }\KeywordTok{mutate}\NormalTok{(}
        \DataTypeTok{sales_text =}\NormalTok{ scales}\OperatorTok{::}\KeywordTok{dollar}\NormalTok{(sales)}
\NormalTok{    ) }\OperatorTok
\StringTok{    }\KeywordTok{ungroup}\NormalTok{() }\OperatorTok
\StringTok{    }\KeywordTok{arrange}\NormalTok{(}\KeywordTok{desc}\NormalTok{(sales)) }\OperatorTok
\StringTok{    }\KeywordTok{set_names}\NormalTok{(}\KeywordTok{c}\NormalTok{(}\StringTok{"Primary Category"}\NormalTok{, }\StringTok{"Sales"}\NormalTok{, }\StringTok{"Sales Text"}\NormalTok{))}
\end{Highlighting}
\end{Shaded}

\begin{verbatim}
## # A tibble: 2 x 3
##   `Primary Category`    Sales `Sales Text`
##   <chr>                 <dbl> <chr>       
## 1 Mountain           39154735 $39,154,735 
## 2 Road               31877595 $31,877,595
\end{verbatim}

Review Secondary Product Category (\texttt{category\_2}).

\begin{Shaded}
\begin{Highlighting}[]
\NormalTok{bike_orderlines_tbl }\OperatorTok
\StringTok{    }\KeywordTok{select}\NormalTok{(category_}\DecValTok{2}\NormalTok{, total_price) }\OperatorTok
\StringTok{    }\KeywordTok{group_by}\NormalTok{(category_}\DecValTok{2}\NormalTok{) }\OperatorTok
\StringTok{    }\KeywordTok{summarise}\NormalTok{(}
        \DataTypeTok{sales =} \KeywordTok{sum}\NormalTok{(total_price)}
\NormalTok{    ) }\OperatorTok
\StringTok{    }\KeywordTok{ungroup}\NormalTok{() }\OperatorTok
\StringTok{    }\KeywordTok{mutate}\NormalTok{(}
        \DataTypeTok{sales_text =}\NormalTok{ scales}\OperatorTok{::}\KeywordTok{dollar}\NormalTok{(sales)}
\NormalTok{    ) }\OperatorTok
\StringTok{    }\KeywordTok{arrange}\NormalTok{(}\KeywordTok{desc}\NormalTok{(sales)) }\OperatorTok
\StringTok{    }\KeywordTok{set_names}\NormalTok{(}\KeywordTok{c}\NormalTok{(}\StringTok{"Type"}\NormalTok{,}\StringTok{"Sales"}\NormalTok{,}\StringTok{"Sales Text"}\NormalTok{))  }
\end{Highlighting}
\end{Shaded}

\begin{verbatim}
## # A tibble: 9 x 3
##   Type                  Sales `Sales Text`
##   <chr>                 <dbl> <chr>       
## 1 Cross Country Race 19224630 $19,224,630 
## 2 Elite Road         15334665 $15,334,665 
## 3 Endurance Road     10381060 $10,381,060 
## 4 Trail               9373460 $9,373,460  
## 5 Over Mountain       7571270 $7,571,270  
## 6 Triathalon          4053750 $4,053,750  
## 7 Cyclocross          2108120 $2,108,120  
## 8 Sport               1932755 $1,932,755  
## 9 Fat Bike            1052620 $1,052,620
\end{verbatim}

Review Frame Material (\texttt{frame\_material}).

\begin{Shaded}
\begin{Highlighting}[]
\NormalTok{bike_orderlines_tbl }\OperatorTok
\StringTok{    }\KeywordTok{select}\NormalTok{(frame_material, total_price) }\OperatorTok
\StringTok{    }\KeywordTok{group_by}\NormalTok{(frame_material) }\OperatorTok
\StringTok{    }\KeywordTok{summarise}\NormalTok{(}\DataTypeTok{sales =} \KeywordTok{sum}\NormalTok{(total_price)) }\OperatorTok
\StringTok{    }\KeywordTok{ungroup}\NormalTok{() }\OperatorTok
\StringTok{    }\KeywordTok{mutate}\NormalTok{(}\DataTypeTok{sales_text =}\NormalTok{ scales}\OperatorTok{::}\KeywordTok{dollar}\NormalTok{(sales)) }\OperatorTok
\StringTok{    }\KeywordTok{arrange}\NormalTok{(}\KeywordTok{desc}\NormalTok{(sales)) }\OperatorTok
\StringTok{    }\KeywordTok{set_names}\NormalTok{(}\KeywordTok{c}\NormalTok{(}\StringTok{"Frame Material"}\NormalTok{,}\StringTok{"Sales"}\NormalTok{,}\StringTok{"Sales Text"}\NormalTok{))}
\end{Highlighting}
\end{Shaded}

\begin{verbatim}
## # A tibble: 2 x 3
##   `Frame Material`    Sales `Sales Text`
##   <chr>               <dbl> <chr>       
## 1 Carbon           52940540 $52,940,540 
## 2 Aluminum         18091790 $18,091,790
\end{verbatim}

\hypertarget{do-all-combinations-primary-and-secondary-bike-category-contain-both-aluminum-and-carbon-frame-materials-difficulty-high}{%
\subsection{3. Do all combinations primary and secondary bike category
contain both Aluminum and Carbon frame materials? (Difficulty =
High)}\label{do-all-combinations-primary-and-secondary-bike-category-contain-both-aluminum-and-carbon-frame-materials-difficulty-high}}

Hint - Use summarized sales values and \texttt{spread()} to identify
gaps in frame materials.

\begin{itemize}
\tightlist
\item
  Select \texttt{category\_1}, \texttt{category\_2},
  \texttt{frame\_material}, and \texttt{total\_price}
\item
  Summarize the data using group by, summarize and ungroup.
\item
  Pivot the frame material and sales column into Alumninum and Carbon
\item
  Fill \texttt{NA} values with zeros
\item
  Add a \texttt{total\_sales} column
\item
  Arrange descending by \texttt{total\_sales}
\item
  Format all numbers as \texttt{dollar()}
\item
  Rename all Columns: Primary Category, Secondary Category, Aluminum,
  Carbon, Total Sales
\end{itemize}

\begin{Shaded}
\begin{Highlighting}[]
\NormalTok{bike_orderlines_tbl }\OperatorTok
\StringTok{    }\KeywordTok{select}\NormalTok{(category_}\DecValTok{1}\NormalTok{, category_}\DecValTok{2}\NormalTok{, frame_material, total_price) }\OperatorTok
\StringTok{    }\KeywordTok{group_by}\NormalTok{(category_}\DecValTok{1}\NormalTok{, category_}\DecValTok{2}\NormalTok{, frame_material) }\OperatorTok
\StringTok{    }\KeywordTok{summarize}\NormalTok{(}
        \DataTypeTok{sales =} \KeywordTok{sum}\NormalTok{(total_price)}
\NormalTok{    ) }\OperatorTok
\StringTok{    }\KeywordTok{ungroup}\NormalTok{() }\OperatorTok
\StringTok{    }\KeywordTok{arrange}\NormalTok{(}\KeywordTok{desc}\NormalTok{(sales)) }\OperatorTok
\StringTok{    }\KeywordTok{mutate}\NormalTok{(}
        \DataTypeTok{sales =}\NormalTok{ scales}\OperatorTok{::}\KeywordTok{dollar}\NormalTok{(sales)}
\NormalTok{    ) }\OperatorTok
\StringTok{    }\KeywordTok{pivot_wider}\NormalTok{(}\DataTypeTok{names_from =}\NormalTok{ frame_material, }\DataTypeTok{values_from =}\NormalTok{ sales) }\OperatorTok
\StringTok{    }\KeywordTok{replace_na}\NormalTok{(}\KeywordTok{list}\NormalTok{(}\DataTypeTok{Carbon =}\NormalTok{ scales}\OperatorTok{::}\KeywordTok{dollar}\NormalTok{(}\DecValTok{0}\NormalTok{), }\DataTypeTok{Aluminum =}\NormalTok{ scales}\OperatorTok{::}\KeywordTok{dollar}\NormalTok{(}\DecValTok{0}\NormalTok{)))}
\end{Highlighting}
\end{Shaded}

\begin{verbatim}
## # A tibble: 9 x 4
##   category_1 category_2         Carbon      Aluminum  
##   <chr>      <chr>              <chr>       <chr>     
## 1 Mountain   Cross Country Race $15,906,070 $3,318,560
## 2 Road       Elite Road         $9,696,870  $5,637,795
## 3 Road       Endurance Road     $8,768,610  $1,612,450
## 4 Mountain   Over Mountain      $7,571,270  $0        
## 5 Mountain   Trail              $4,835,850  $4,537,610
## 6 Road       Triathalon         $4,053,750  $0        
## 7 Road       Cyclocross         $2,108,120  $0        
## 8 Mountain   Sport              $0          $1,932,755
## 9 Mountain   Fat Bike           $0          $1,052,620
\end{verbatim}


\end{document}
